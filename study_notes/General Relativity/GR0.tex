\documentclass[fontset=windows]{article}
\usepackage[margin=1in]{geometry}%设置边距,符合Word设定
\usepackage{ctex}
\usepackage{setspace}
\usepackage{lipsum}
\usepackage{graphicx}%插入图片
\usepackage{amsmath}

\usepackage{hyperref} % 对目录生成链接,注:该宏包可能与其他宏包冲突,故放在所有引用的宏包之后
\hypersetup{colorlinks = true,  % 将链接文字带颜色
	bookmarksopen = true, % 展开书签
	bookmarksnumbered = true, % 书签带章节编号
	pdftitle = This is a testfile for vscode, % 标题
	pdfauthor =Ali-loner} % 作者
\bibliographystyle{plain}% 参考文献引用格式
\newcommand{\upcite}[1]{\textsuperscript{\cite{#1}}}

\renewcommand{\contentsname}{\centerline{Contents}} %经过设置word格式后,将目录标题居中


\title{\heiti\zihao{2} 广义相对论学习笔记(day0)}
\author{\songti pupllen}
\date{\today}

\begin{document}
	\maketitle
我们首先从数学上,从狭义相对论的条件中引入定义。

首先定义的相对论中的一个几何概念(亦即研究对象)是时空,即数学定义为:

Spacetime is a manifold of events that endowed with a  metric.

时空是事件有度规的流形

这个定义又由3个概念组成,它们分别定义为:

流形(manifold): A set of points with well-understood connectedness properties.

一组符合某种物理前提的点集,有良好的连通性。

事件(events): Where and when something happens, usually we label it with coordinates.

时空中的点,一般写为一组坐标。

度规(metric): A notion of distance between events.

描述时空中点之间的几何关系(通常是距离)的物理量。

\hspace*{\fill}

有了这些基本概念,我们接下来便讨论最简单的一种情况:没有引力的时空。

为了方便讨论,接下来我们只用正交(orthogonal)的坐标系。

这个时空的事件可以写为一组四维坐标,即
$(t,x,y,z)$,表示时间的$t$与位置的三维向量$\vec{x} = (x,y,z)$。

\hspace*{\fill}

现在我们假设时空中有两个事件,即$(t_1,x_1,y_1,z_1)$和$(t_2,x_2,y_2,z_2)$。%其中$\vec{x_1} \neq \vec{x_2}$

假设在某个时刻,事件1的位置向事件2的位置发射了一束光,在位置2接收到这束光时,又将其反射回位置1。记发射时$t_1 = t_{1e}$,接收到反射光时$t_1 = t_{1r}$,而位置2反射时为$t_2 = t_{2b}$。由于我们讨论的是没有引力的情况,因此在惯性参考系下(inertial frame)
狭义相对论的光速不变假设意味着,如果一开始时且有$t_1 = t_2 = 0$,则下式成立:
\begin{equation}
	t_{2b} = \frac{1}{2}(t_{1e}+t_{1r})
\end{equation}
通过这样的方式,我们实际上是校准了2个时钟。

有了时间,配合上原本一般坐标系的表示距离的方式,我们可以定义事件关于时空的距离(spacetime interval):
\begin{equation}
	s^2 = -(c\Delta t)^2 + (\Delta x)^2 + (\Delta y)^2 + (\Delta z)^2
\end{equation}
我们会看到,其中$c$为光速这件事和时空距离在惯性坐标系中不变是等价的,或者更一般地说,存在一个将时间和空间相互转换的固定因子,当其为光速时,时空距离在惯性坐标系中不变。

现在需要引入一些方便的记号,也是之后广义相对论主要使用的,记:
\begin{equation*}
	x^\mu  \Rightarrow \mu \in (t,x,y,z) \,\text{or}\, (0,1,2,3)
\end{equation*}
且
\begin{equation*}
	c = 1
\end{equation*}
因此我们可以写出一种表示时空距离更紧凑的形式,即利用矩阵(度规)的形式,引入:
\begin{equation}
	\eta_{\mu\nu} = 
	\left(
	\begin{array}{cccc}
		-1 & 0 & 0 & 0 \\
		0 & 1 & 0 & 0 \\
		0 & 0 & 1 & 0 \\
		0 & 0 & 0 & 1 \\
	\end{array}
	\right)
\end{equation}
那么时空距离就可以写为:
\begin{equation}
	s^2 = \eta_{\mu\nu}\Delta x^{\mu}\Delta x^{\nu}
\end{equation}
这里使用了爱因斯坦约定求和:对重复出现的指标时,式子自动对指标依次求和。



\end{document}
