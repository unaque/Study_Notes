\documentclass[fontset=windows]{article}
\usepackage[margin=1in]{geometry}%设置边距,符合Word设定
\usepackage{ctex}
\usepackage{setspace}
\usepackage{lipsum}
\usepackage{graphicx}%插入图片
\usepackage{amsmath}

\usepackage{hyperref} % 对目录生成链接,注:该宏包可能与其他宏包冲突,故放在所有引用的宏包之后
\hypersetup{colorlinks = true,  % 将链接文字带颜色
	bookmarksopen = true, % 展开书签
	bookmarksnumbered = true, % 书签带章节编号
	pdftitle = This is a testfile for vscode, % 标题
	pdfauthor =Ali-loner} % 作者
\bibliographystyle{plain}% 参考文献引用格式
\newcommand{\upcite}[1]{\textsuperscript{\cite{#1}}}

\renewcommand{\contentsname}{\centerline{Contents}} %经过设置word格式后,将目录标题居中


\title{\heiti\zihao{2} 计算机网络学习笔记(day1)}
\author{\songti pupllen}
\date{\today}

\begin{document}
	\maketitle
	如果从网络组成类型的角度,互联网由3个类型组成:网络核心(core),网络边缘(edge)和接入系统(access)。

	网络边缘一般指可以联网的终端,例如电脑主机,应用程序等等。之所以称为边缘,是因为真正的核心应该是将所有这些可以联网的设备链接在一起的部分。如果一上来就将任何两个设备直接的链接在一起,那将是十分浪费且臃肿的系统。所以,要实现所有设备的联网,应当先把这些设备链接到一个核心,然后再利用核心实现任意两个设备的链接。
	而接入系统指的就是将网络边缘接入到网络核心的部分。

	网络边缘之间通信的模式有两种:客户端/服务器模式(client-sever mode)和对等模式(peer-peer,p2p)。
	在cs模式中,分布式应用分为主和从,客户端为从,服务器为主,是相当常见的模式,例如各种联网游戏。根据客户端的动作向服务器发出请求,服务器处理完成请求后再将指令发送返回客户端使其完成结果动作。而在p2p模式中,没有典型的集成的服务器,例如许多网络下载器不是专门用一个大型服务器储存资源,而是每个下载资源的用户通过链接在一起,完成了资源的集合。亦即,网络中每个分布式应用既是客户端也是服务端,。

	采用tcp协议的网络边缘意味着,该网络向上层提供面向链接的服务,两个通信主机之间为链接的建立做准备。这样的服务是可靠的,能够做到不重复,不失序,不出错,不丢失。且tcp协议实体能够提供流量控制和拥塞控制的服务。例如FTP传输文件,HTTP等采用的就是tcp协议。

	而采用udp协议的网络边缘便没有上述特性,称为无连接的服务。例如流媒体应用,视频通话等等多采用udp协议。

	网络核心一般是由网络交换设备组成的网络,网络交换设备则一般是路由器。关于网络核心基本的问题就是:如何将实现通信设备的链接。例如,在端到端已有的线路之间,建立起一个独享的线路(称为线路中的piece),称为从端到端的呼叫(call),这种链接方式称为电路交换,一般是电话这样设备采用这种链接方式。

	那么关于电路交换,问题又是:如何将已有线路分为一个个独立的部分(piece)。有3种典型方式:频分(FDM),时分(TDM),波分(WDM),举个例子说明实现电路交换连接方式的网络如何传输信息。

	在一个电路交换的网络上,从主机A到主机B发送大小$640000 bit$的文件需要多少时间
	
	已知,主机A与主机B之间所有链路的总速率为$1.536Mbps$,每条链路通过TDM分为24的时隙数,端与端之间的独立链接建立需要$500ms$。

	关键在于知道,电路交换从总链路中建立独立链路的3种方式中,都是均分的。因此,A和B之间的实际传输速率(不考虑物理层的损耗)是
	\begin{equation}
		v_{t} = \frac{1.536Mbps}{24} = 64Kbps 
	\end{equation}

	即传输时间
	\begin{equation}
		t = \frac{640Kbits}{64Kps}  = 10s 
	\end{equation}

	因此,主机A文件发送出去的时间是$10.5s$。然而,主机B完成接收文件的时间还要更长一些,这是因为还要考虑到文件在物理层的运输时间和打包时间(每个bit准备发送的时间),这一时间通常和主机间的距离,光速和带宽相关。

	电路交换并不适合计算机间的通信,因为其建立链接的时间长,且难以对应突发的计算机通信。

	因此更常用的链接方式是分组交换。主机A将传输的数据本身分组(即分为包packet),然后打包发给中转站临时储存,再由中转站发给主机B,直到所有包完成发送。此时传输使用链路的全部的带宽,不传输时就不用带宽,而不会出现电路交换中建立了独立线路而没有传输,从而浪费网络资源的情况。在这种情况下存在储存-转发的延迟:
	\begin{equation}
		t_{delay} = n*\frac{L}{R}
	\end{equation}
	其中$R$指链路的传输速率,$L$是文件的大小,$n$是储存-转发的次数,表示A到B需要经过多少次中转才能到达(多少跳)。

	在分组交换中,如果许多主机都要通过一个线路,那么还会有排队延迟,其分组队列会等待前面的包发送完成才进行发送,这种情况下分组可能丢失。

	你可以看到,分组交换的关键在于:确定传输的路径。这就是路由的名字来源,根据路由算法得到传输路径从而建立链接。
	
	同样的条件下,分组交换相比电路交换能支持更多的用户。例如,现有总线路带宽为$1Mbps$,当线路活跃时,每个用户占用$100Kbps$的带宽,线路活跃的概率是$p = 0.1$

	在电路交换的情况下,能够支持的用户显然是:
	\begin{equation}
		n = \frac{1Mbps}{100Kbps} = 10
	\end{equation}
	
	在分组交换的情况下,当我们取用户为N,则有10个以上用户活跃的概率是
	\begin{equation}
		P = 1 - \sum_{0}^{10}\binom{N}{n}p^n(1-p)^{N-n}
	\end{equation}
	使用柏松近似计算这个概率,结果为
	\begin{equation}
		P = 1 - 
	\end{equation}

	在分组交换中,按照有无网络层的链接,分为:数据报网络和虚电路网络。

\end{document}