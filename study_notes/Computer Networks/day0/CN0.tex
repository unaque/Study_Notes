\documentclass[fontset=windows]{article}
\usepackage[margin=1in]{geometry}%设置边距,符合Word设定
\usepackage{ctex}
\usepackage{setspace}
\usepackage{lipsum}
\usepackage{graphicx}%插入图片

\usepackage{hyperref} % 对目录生成链接,注:该宏包可能与其他宏包冲突,故放在所有引用的宏包之后
\hypersetup{colorlinks = true,  % 将链接文字带颜色
	bookmarksopen = true, % 展开书签
	bookmarksnumbered = true, % 书签带章节编号
	pdftitle = This is a testfile for vscode, % 标题
	pdfauthor =Ali-loner} % 作者
\bibliographystyle{plain}% 参考文献引用格式
\newcommand{\upcite}[1]{\textsuperscript{\cite{#1}}}

\renewcommand{\contentsname}{\centerline{Contents}} %经过设置word格式后,将目录标题居中


\title{\heiti\zihao{2} 计算机网络学习笔记(day0)}
\author{\songti pupllen}
\date{\today}

\begin{document}
	\maketitle
	互联网指的是计算机的网络和网络之间的链接,计算机网络则是由层次结构而成的,称为

	开放式系统互联模型(Open System Interconnection Model)简称OSI模型。
	
	从上至下依次是:
	
	应用层(Application)
	
	表示层(Presentation)
	
	会话层(Session)
	
	传输层(Transport)
	
	网络层(Network)
	
	数据链路层(Data Link)
	
	物理层(Physical)
	
	第一节:
	
	什么是互联网,首先要讲讲什么是网络。
	
	简单从数学上讲,由节点和边组成的图网即可称为网络,而具体到计算机网络中,有多种节点和边,以及关于这些节点和边的“标准”,即协议(protocol)。例如,节点可以代表主机及其运行的应用程序,或者路由器、交换机等网络交换设备,而边可以是主机链接到互联网的链路(接入网链路),抑或是主干路链路(路由器间的链路)等等,而协议要看具体是网络的那一层,不同层是相互独立的,因此协议分成几个层次进行定义。
	
	有了网络的基本概念,互联网的概念便顺应而生。
	
	概括地讲,互联网即是,一簇以TCP和RCP为主的协议支撑其工作的网络。
	
	从具体构成的方面来说,互联网往往由:计算设备,通信链路和分组交换设备组成,由协议控制消息的收发。由一堆网络由网络交换设备链接在一起(网络的网络)。
	
	而互联网的协议是由RFC文档(Request for comments)发展而来,这些文档会发送到一个称为IETF(Internet Engineering Task Force)的地方并编号,供全球所有工程师,科学家等等访问。
	
	网络协议一般包括如下规范:语法,语义,时序,动作。(报文,次序,动作)
	
	从用户服务的角度来说,计算机网络由分布式应用以及通信基础设施组成,分布式应用利用基础设施进行通信,基础设施将分布式应用链接起来,并提供网络服务。亦即,基础设施为apps提供网络服务的api。
	
	例如,如果基础设施的协议是UDP协议(User Datagram Protocol),那么这个网络的服务可以称为无连接不可靠服务,其优点是开销小,而如果是TCP协议(Transmission Control Protocol),则称为面向链接的可靠服务,其开销较大。
    
\end{document}